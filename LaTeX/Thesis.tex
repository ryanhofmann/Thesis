%% Format in ApJ style
\documentclass[iop]{emulateapj}

\newcommand{\myemail}{ryanhofmann@email.arizona.edu}

\begin{document}

%% Title
\title{Observations of the Type IIn SN J13522411}

%% Authors and affiliations
\author{Ryan Hofmann\altaffilmark{1} and Nathan Smith\altaffilmark{2}}
\affil{14947 E. Sierra Madre Dr, Fountain Hills, AZ 85268, USA}
\affil{Steward Observatory, University of Arizona, 933 N. Cherry Ave, Tucson, AZ 85721, USA}

%% Abstract
\begin{abstract}
We present visible and near-IR photometry and spectra of the type IIn supernova PSN J13522411. The light curve, though poorly sampled, indicates a peak $<50$ days after discovery around $M = -17$, followed by a slow decline. Spectra show strong hydrogen and helium emission that peaks $\sim200$ d after discovery, with a P-Cygni profile that becomes more absorptive as time progresses. We estimate the CSM has a radius of $\sim90$ AU and was ejected $<12$ yr before the explosion, with the velocity of the CSM implying a red or yellow supergiant as the progenitor. From the light curve, we estimate the total radiated energy to be $\sim10^{50}$ erg. Further details of the explosion will be determined as time allows.
\end{abstract}

%% Introduction
\section{Introduction}
Understanding the explosive deaths of massive stars as supernovae is an important area of astrophysical research, as it is through this process that heavy elements are formed and dispersed into the interstellar medium. One of the more interesting aspects of these phenomena, and one that is still not well understood, is how supernovae interact with their environments. A striking example of this interaction is the class SNe IIn, which have narrow emission lines and slowly declining light curves \citep{Fil97}.

Most of the peculiar features of SNe IIn can be explained by a shell of dense circumstellar material ejected from the star before it exploded. Initially, intense UV- and X-ray radiation from the explosion photoionize the slow-moving CSM, producing narrow-line emission with a Lorentzian profile. Later, the rapidly expanding shock collides with the CSM, causing broad emission with a line profile that depends on the distribution of the CSM.

In this paper, we present visual and near-IR photometry and spectroscopy of PSN J13522411 from discovery up to $\sim450$ d afterward. PSN J13522411 was discovered by Zhangwei Jin and Xing Gao on 2015 Jan. 14.9 in three unfiltered images of NGC 5337. The reported apparent magnitude was 16.9, which we assume to be equivalent to the R magnitude, but is more likely to be an upper limit. No source was visible in images taken on Jan. 07 (limiting mag 19.5). It was classified as having a type IIn spectrum on 2015 Jan. 16.9 by Jujia Zhang and Xiaofeng Wang. At a redshift of $z = 0.007007$ (2165 km/s), this corresponds to a distance of 36.39 Mpc ($m - M = 32.8$ mag), with a Milky Way line-of-sight extinction $A_v = 0.038$ mag and reddening $E(B - V) = 0.013$ mag. PSN J13522411 resides in the outer disk of NGC 5337, at a projected separation of $\sim18$ arcsec ($\sim3$ kpc) from the host galaxy's nucleus. Our observations are presented in Section 2, and the light curve and spectral evolution are analysed in Section 3. Section 4 provides a summary of our work.

%% Observations
\section{Observations}
\subsection{Discovery}
PSN J13522 was discovered on three 40-s survey images taken by Xing Gao in Xingming Observatory with an unfiltered CCD on a Celestron C14 Schmidt-Cassegrain telescope. The object first appeared on Jan. 09.9, and was confirmed on Jan. 14.9; nothing was visible in images taken on Jan. 07 (limiting mag 19.5).

\subsection{SLOTIS photometry}
After discovery of PSN J13522, the field was added to the queue of the robotic Super-LOTIS 24-inch telescope \citep[SLOTIS;][]{Wil08} on Kitt Peak for multifilter (B, V, R, and I) follow-up observations. Seeing varied between $\sim2-4$ arcsec. Images were automatically calibrated using a custom pipeline by Peter Milne, and aperture photometry was performed manually. The magnitudes were calibrated using the reference star list from the SLOTIS pipeline, shown in Table 4. The photometry is summarized in Table 1.

\subsection{UKIRT JHK photometry}
Three sets of JHK images were collected during the same period as the SLOTIS data, using the UK Infrared Telescope's (UKIRT) Wide Field Camera instrument \citep[WFCAM;][]{Hod09}. The 'seeing', estimated from the full width at half-maximum intensity (FWHM) of stars on the CCD frame, varied between $\sim1-2$ arcsec. Aperture photometry was performed manually, and the magnitudes were calibrated using the same reference stars from SLOTIS. The results are summarized in Table 2.

\subsection{Kuiper BVR photometry}
One set of images was recorded at a late time using the Mont4k CCD on the 61" Kuiper telescope on Mt. Bigelow \citep{Fon14}. Seeing was $<0.5$ arcsec. Aperture photometry was performed manually. Some of the reference stars were outside the field of view or saturated, and were thus excluded from the reduction. The results are summarized in Table 3.

\subsection{Spectroscopy}
Five high-resolution spectra were obtained using the Bluechannel (BC) spectrometer on the MMT with the 1200 line grating. Four of the spectra were taken during the first six months, while the fifth was taken much later. One early spectrum was obtained using the Multi Object Double Spectrograph \citep[MODS]{Bya00} on the LBT. Two broad spectra were obtained using the Kast spectrograph on the Lick 3-m Shane reflector \citep{Mil93}. Finally, one late-time spectrum was obtained using the Boller \& Chivens (B\&C) spectrograph on the Bok 90-inch telescope on Kitt Peak. All spectra were Doppler-corrected, and the broad spectra were also corrected for reddening. Details of the spectra are summarized in Table 4.

%% Analysis
\section{Analysis}
\subsection{Light curve}
\subsubsection{General features}
The multiband light curves of PSN J13522411 are shown in Fig. 2. No data is available around the time of peak luminosity, so the exact peak date could be anywhere between $\sim10-50$ days. Likewise, the later portion of the light curve has a $>250$ d gap where no data was taken, so we can only assume that the decline during this period is basically linear. Despite the lack of sampling, it is clear that the SN is decaying quite slowly: in $\sim300$ days, the R brightness only decreased by $\sim1.1$ mag.

\subsubsection{Total emitted energy}
If we assume that the R-band brightness is roughly equivalent to the bolometric magnitude, we can transform the R magnitudes into luminosities and integrate over time to get an estimate of the total radiated energy. Performing this calculation \citep{Are99}, we estimate the total emitted energy to be $\sim4 \times 10^{49}$ erg. Because we were not able to obtain any data during the period around maximum brightness, this value is probably a lower limit, putting the true value closer to $\sim10^{50}$ erg.

\subsection{Spectral evolution}
\subsubsection{General properties}
Besides the H-alpha line discussed below, the spectra from Lick Observatory also show H-beta emission (weakly visible in Bok), bright emission lines in the near-infrared around $\sim8500$ angstroms, and broad helium emission lines. The helium line at 5876 angstroms is bisected by strong absorption from the sodium D doublet. The H-alpha line also features several small absorption lines between $\sim6450-6525$ angstroms.

\subsubsection{Continuum}
Initially, the continuum of PSN J13522411 was flat over the range of the MMT, while the LBT spectrum showed emission and absorption features in the near-infrared. Over the next few months, the continuum became more red, as seen in both the MMT and Lick spectra. This trend reversed sometime around 400 days after discovery, as the last MMT spectrum seems less diagonal, while the final spectrum from Bok has a blue continuum with a clear peak around $\sim5000$ angstroms, implying an effective temperature of $\sim5800$ K.

\subsubsection{H-alpha line evolution}
Initially, the broad H-alpha line is symmetrical with a FWHM of $\sim1400$ km/s, with Lorentzian wings extending out to $\sim\pm5000$ km/s. On top of the broad component is a narrow P-Cygni feature, predominately emission, with a Gaussian FWHM of $\sim50$ km/s, about the resolution of the grating. As time progresses, the broad emission increases and becomes more asymmetrical and humped until $\sim200$ d after discovery, after which the H-alpha flux starts to decrease. The P-Cygni feature gradually transitions from mostly emission to almost all absorption, causing the appearance of a double peak in the broad emission. The FWHM of the broad emission rises rapidly, peaks around $\sim200$ d after discovery, then declines more gradually. The peak-to-trough width of the P-Cygni feature remains roughly constant at $\sim70$ km/s, just over the resolution of the grating.

\subsubsection{CSM properties}
Taking the half-width of the broad H-alpha emission as representative of the shock velocity, we estimate the radius of the CSM to be $\sim80-90$ AU. If we assume that the half-width of the narrow emission line indicates the velocity of the CSM, we arrive at an upper age limit of $\sim10$ years. The low velocity of the pre-SN material indicates that the progenitor was most likely a red or yellow supergiant \citep{Smi15}, rather than a Wolf-Rayet star or luminous blue variable (LBV).

%% Conclusion
\section{Conclusion}
We obtained photometry and spectra of PSN J13522411 covering the early and later portions of the SN's decline. By integrating the light curve, we estimate the total radiated energy at $\sim10^{50}$ erg. From the spectral data, we infer an age for the CSM of less than 12 years. We also infer from the velocity of the stellar wind that the progenitor was likely a red or yellow supergiant. Due to time constraints, we have not yet completed our analysis of this SN. In the coming weeks, we plan to determine the mass and distribution of the CSM and the total energy of the explosion.

%% Acknowledgements
\section{Acknowledgements}
Special thanks goes to Dr. Jennifer Andrews for her assistance with data management and reduction. Also to Dr. Peter Milne, Principal Investigator for the Super-LOTIS robotic telescope, for his advice on working with the SLOTIS data.

%% Bibliography
\begin{thebibliography}{}
\bibitem[Aretxaga et al.(1999)]{Are99} Aretxaga I. et al., 1999, MNRAS, 309, 343
\bibitem[Byard and O'Brien(2000)]{Bya00} Byard P. L., O'Brien T. P., 2000, Proc. SPIE, 4008, 934
\bibitem[Filippenko(1997)]{Fil97} Filippenko A. V., 1997, ARA\&A, 35, 309
\bibitem[Fontaine et al.(2014)]{Fon14} Fontaine G. et al., 2014, ASP Conference Series, 481, 19
\bibitem[Hodgkin et al.(2009)]{Hod09} Hodgkin S. T. et al., 2009, MNRAS 394, 675
\bibitem[Miller and Stone(1993)]{Mil93} Miller J. S., Stone R. P. S., 1993, Lick Obs. Tech. Rep. 66, Lick Obs., Santa Cruz
\bibitem[Smith et al.(2015)]{Smi15} Smith N. et al., 2015, MNRAS, 449, 1876
\bibitem[Williams et al.(2008)]{Wil08} Williams G. G. et al., 2008, AIP Conference Proceedings, 1000, 535
\end{thebibliography}

\end{document}
